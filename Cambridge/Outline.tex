\documentclass[12pt, a4paper, oneside]{hitec}\usepackage[]{graphicx}\usepackage[]{color}
%% maxwidth is the original width if it is less than linewidth
%% otherwise use linewidth (to make sure the graphics do not exceed the margin)
\makeatletter
\def\maxwidth{ %
  \ifdim\Gin@nat@width>\linewidth
    \linewidth
  \else
    \Gin@nat@width
  \fi
}
\makeatother

\definecolor{fgcolor}{rgb}{0.345, 0.345, 0.345}
\newcommand{\hlnum}[1]{\textcolor[rgb]{0.686,0.059,0.569}{#1}}%
\newcommand{\hlstr}[1]{\textcolor[rgb]{0.192,0.494,0.8}{#1}}%
\newcommand{\hlcom}[1]{\textcolor[rgb]{0.678,0.584,0.686}{\textit{#1}}}%
\newcommand{\hlopt}[1]{\textcolor[rgb]{0,0,0}{#1}}%
\newcommand{\hlstd}[1]{\textcolor[rgb]{0.345,0.345,0.345}{#1}}%
\newcommand{\hlkwa}[1]{\textcolor[rgb]{0.161,0.373,0.58}{\textbf{#1}}}%
\newcommand{\hlkwb}[1]{\textcolor[rgb]{0.69,0.353,0.396}{#1}}%
\newcommand{\hlkwc}[1]{\textcolor[rgb]{0.333,0.667,0.333}{#1}}%
\newcommand{\hlkwd}[1]{\textcolor[rgb]{0.737,0.353,0.396}{\textbf{#1}}}%
\let\hlipl\hlkwb

\usepackage{framed}
\makeatletter
\newenvironment{kframe}{%
 \def\at@end@of@kframe{}%
 \ifinner\ifhmode%
  \def\at@end@of@kframe{\end{minipage}}%
  \begin{minipage}{\columnwidth}%
 \fi\fi%
 \def\FrameCommand##1{\hskip\@totalleftmargin \hskip-\fboxsep
 \colorbox{shadecolor}{##1}\hskip-\fboxsep
     % There is no \\@totalrightmargin, so:
     \hskip-\linewidth \hskip-\@totalleftmargin \hskip\columnwidth}%
 \MakeFramed {\advance\hsize-\width
   \@totalleftmargin\z@ \linewidth\hsize
   \@setminipage}}%
 {\par\unskip\endMakeFramed%
 \at@end@of@kframe}
\makeatother

\definecolor{shadecolor}{rgb}{.97, .97, .97}
\definecolor{messagecolor}{rgb}{0, 0, 0}
\definecolor{warningcolor}{rgb}{1, 0, 1}
\definecolor{errorcolor}{rgb}{1, 0, 0}
\newenvironment{knitrout}{}{} % an empty environment to be redefined in TeX

\usepackage{alltt} % Paper size, default font size and one-sided paper
%\usepackage[dcucite]{harvard}
\usepackage{graphicx}
%\bibliographystyle{plainnat}
\bibliographystyle{agsm}
\usepackage[colorlinks = true, citecolor = blue, linkcolor = blue]{hyperref}
%\hypersetup{urlcolor=blue, colorlinks=true} % Colors hyperlinks in blue - change to black if annoying
%\renewcommand[\harvardurl]{URL: \url}
\usepackage{listings}
\usepackage{color}
\graphicspath{{../Pictures/}}
\definecolor{mygrey}{gray}{0.95}
\lstset{backgroundcolor=\color{mygrey}}
\IfFileExists{upquote.sty}{\usepackage{upquote}}{}
\begin{document}
\title{Trading and Financial Markets}
\author{Khaled Soufani and Rob Hayward}
%\date{\today}
\maketitle
%\begin{abstract}
%\end{abstract}
\section*{Introduction}
This unit aims to provide participants with an introduction to financial markets and the way that securities are traded.  It presents a method of evaluating trading success and introduces some common trading strategies.  The unit is composed of four parts that culminate in the presentation of an investment strategy.  The unit will require some pre-reading activity.

\section*{Part One: Introduction to financial markets}
\emph{Rotman Interactive Trader} and the Microstructure Case Studies will be used to introduce electronic markets, market and limit orders as well as trading costs, margin and the bid-ask spread. Traders will have to manage customer orders and make decisions about their use and provision of liquidity.  
\begin{itemize}
\item Microstructure, the organisation of markets and the evolution of these institutions. 
\begin{itemize}
\item open-outcry
\item dealer/market-maker
\item electronic orders
\end{itemize}
\item The key market participants, their role and motivation. 
\begin{itemize}
\item market-makers
\item speculators
\item noise or liquidity traders
\item real money and long-term accounts
\end{itemize}
\item The importance of liquidity: avoiding \emph{fat fingers} and getting paid for providing liquidity
\end{itemize}

\section*{Part Two: Information and price discovery}
The price discovery process, market efficiency and inefficiency.  Information, speculation and sentiment. \emph{Rotman Interactive Trader} price discovery case studies reveal intrinsic value.  The relationship between the cost of acquiring information and the benefit that can be gained from it. 
\begin{itemize}
\item The efficient market hypothesis and the moving spectrum of efficiency
\item Behavioural finance and \emph{systematic mistakes}
\item Institutional causes of inefficiency
\end{itemize}

\section*{Part Three: Technical trading rules and event studies}
There are two standard ways of identifying market inefficiencies: technical trading rules and event studies.  Technical rules seek to identify trends and turning points in financial markets; event studies assess the effect of particular events on securities. Each of these are used to try to find market inefficiencies that can be exploited. 
\begin{itemize}
\item An introduction to technical analysis: `the trend is your friend` and turning points
\item Technical trading rules: back-testing. 
\item Event studies: does the outcome differ from the efficient reaction, can it be exploited?
\end{itemize}

\section*{Part Four: Other trading ideas}
The unit concludes by presenting a number of other commonly used trading techniques and with an invitation for participants to develop their own ideas using the framework that has been outlined in the unit. 
\begin{itemize}
\item Paris-trading to isolate risk
\item ETFs for dynamic asset-allocation
\item Carry-trades across exchange rates and interest rates
\item Applying value-investment to other fields and time frames
\end{itemize}


\end{document}
